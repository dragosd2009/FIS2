\documentclass{article}
\usepackage[utf8]{inputenc}
\usepackage{enumerate}
\usepackage{amssymb}
\usepackage{graphicx}

\title{Proiect SDA}
\author{Dragos-Gabriel Danciulescu}
\date{May 2022}


\begin{document}

\maketitle

\section*{Example Exercise}

Use the baby-step giant-step method to find x such that   $$ 5^{x} \equiv 107 \bmod{179} $$. Pick m = $\lfloor\sqrt{89}\rfloor$.Copy, extend and fill the
appropriate tables.\\
\begin{enumerate}[a.]
    \item
We will create the Baby steps table in which $g=5$ for $i$ up to $9$:\\
\begin{tabular}{ |c|c|c|c|c|c|c|c|c|c|c| }
 \hline
 i & 0 & 1 & 2 & 3 & 4 & 5 & 6 & 7 & 8 & 9 \\
 \hline
 $5^i \bmod 179$ & 1 & 5 & 25 & 125 & 88 & 82 & 52 & 81 & 47 &  56\\
 \hline

\end{tabular}
\\
Next, we note that:\\
$g^{-\sqrt{89}}=5^{-9}=56^{-1}=16 \bmod 179$\\
So, $h \cdot g^{-mj} = 107 \cdot 16^j$\\
Now,we will create the Giant steps table. Note that in the algorithm, these values are not stored in memory like the ones in the first table.\\
\begin{tabular}{ |c|c|c|c| }
 \hline
 j & 0 & 1 & 2\\
 \hline
 $107 \cdot 16^j \bmod 179$ & 107 & 101 & 5\\
 \hline

\end{tabular}
We can now observe that we've found a match, for $i=1$ and $j=2$.\\
So,following the theory:\\
$g^{i+m \cdot j}=h$\\
$5^{1+9 \cdot 2}=107$\\
$5^{19} \bmod 179 = 107$\\
We've also checked the value with Wolfram Alpha, it is correct, so:\\
$x=19$\\

\end{enumerate}


\section*{Notebook Examples and Theory}
\begin{figure}
  \includegraphics[width=\textwidth]{GCD.png}
  \caption{Extended Euclidean algorithm notebook example}
  \label{fig:D_P}
 \end{figure}

\begin{figure}
  \includegraphics[width=\textwidth]{BSGS.png}
  \caption{Baby-Step Giant-Step notebook example}
  \label{fig:D_P}
\end{figure}

\begin{figure}
  \includegraphics[width=\linewidth]{theory_gcd.png}
  \caption{Greatest Common divisor}
  \label{fig:D_P}
\end{figure}

\begin{figure}
  \includegraphics[width=\linewidth]{euclidean.png}
  \caption{Euclidean algorithm}
  \label{fig:D_P}
\end{figure}

\begin{figure}
  \includegraphics[width=\linewidth]{EEA.png}
  \caption{Extended euclidean algorithm}
  \label{fig:D_P}
\end{figure}

\begin{figure}
  \includegraphics[width=\linewidth]{Invertibility.png}
  \caption{Computing inverse using Euclid's algorithm}
  \label{fig:D_P}
\end{figure}

\begin{figure}
  \includegraphics[width=\linewidth]{DiscreteLog.png}
  \caption{The discrete log problem in cryptography}
  \label{fig:D_P}
\end{figure}

\begin{figure}
  \includegraphics[width=\linewidth]{TheoryBSGS.png}
  \caption{Explanation of Baby-Step Giant-Step algorithm}
  \label{fig:D_P}
\end{figure}

\begin{figure}
  \includegraphics[width=\linewidth]{Discussion.png}
  \caption{A discussion about BSGS}
  \label{fig:D_P}
\end{figure}





\end{document}
